% Copyright Luke Olson 2009--2014
% This work is licensed under the Creative Commons
% Attribution-NonCommercial-NoDerivatives 4.0 International License. To view a
% copy of this license, visit http://creativecommons.org/licenses/by-nc-nd/4.0/.
%
\documentclass[10pt]{beamer}
%\documentclass[handout,10pt]{beamer}
%pdflatex -jobname lecture15.print lecture15
%
\mode<presentation>
{
  \usetheme[secheader]{Boadilla}
  \usefonttheme[onlymath]{serif}
  \setbeamercovered{invisible}
  \usecolortheme{luke}
  %\setbeamercovered{transparent}
  %
}
\mode<handout>
{
  \usetheme[secheader]{Boadilla}
  \usefonttheme[onlymath]{serif}
  \setbeamercovered{invisible}
  \usecolortheme{dove}
  %\setbeamercovered{transparent}
}
\usepackage{pgf,pgfarrows,pgfnodes,pgfautomata,pgfheaps,pgfshade}
\usepackage{pxfonts}
\usepackage{eulervm}
\usepackage{listings}
%\usepackage{pgfpages}
%\pgfpagesuselayout{2 on 1}[letterpaper]
%
%
%%%%%%%%%%%%%%%%%%%%%%%%%%%%%%%%%%%%%%%%%%%%%%%%%%%%%%%%%%%%%%%%%%%%%%%%


%
%
%
\newcommand{\vb}{{\bf{b}}}
\newcommand{\ve}{{\bf{e}}}
\newcommand{\vg}{{\bf{g}}}
\newcommand{\vp}{{\bf{p}}}
\newcommand{\vr}{{\bf{r}}}
\newcommand{\vu}{{\bf{u}}}
\newcommand{\vx}{{\bf{x}}}
\newcommand{\vz}{{\bf{z}}}
\newcommand{\vA}{{\bf{A}}}
\newcommand{\vU}{{\bf{U}}}
\newcommand{\mO}{{\mathcal{O}}}
\newcommand{\mF}{{\mathcal{F}}}
\definecolor{mygray}{rgb}{0.95,0.95,0.95}
\lstset{
        language=matlab,
        numbers=left, numberstyle=\tiny, stepnumber=1, numbersep=5pt,
        basicstyle=\color{black}\ttfamily\small,
        commentstyle=\color{green}\ttfamily,
        keywordstyle=\color{blue}\ttfamily,
        stringstyle=\color{red}\ttfamily,
        showstringspaces=false,
        backgroundcolor=\color{mygray},
        breaklines,
}
\newcommand{\norm}[1]{{\ensuremath{{\|#1\|}}}}
\newcommand{\matdim}[2]{\ensuremath{#1\times#2}}
\newcommand{\rank}[1]{\ensuremath{\mathrm{rank}(#1)}}
\newcommand{\epsm}{\ensuremath{\varepsilon_m}}
\newcommand{\cmd}[1]{{\normalfont\ttfamily\bfseries#1}}

\author{L. Olson}
\institute[UIUC]
{Department of Computer Science\\
University of Illinois at Urbana-Champaign\\
\vspace{0.5cm}
}
%%%%%%%%%%%%%%%%%%%%%%%%%%%%%%%%%%%%%%%%%%%%%%%%%%%%%%%%%%%%%%%%%%%%%%%%
\pgfdeclareimage[height=0.5cm]{university-logo}{./figs/uiuclogo}
\logo{\pgfuseimage{university-logo}}
%%%%%%%%%%%%%%%%%%%%%%%%%%%%%%%%%%%%%%%%%%%%%%%%%%%%%%%%%%%%%%%%%%%%%%%%
\title[CS 357]{Lecture 24}
\subtitle{Monte Carlo}
\date{November 19, 2009}

\begin{document}
% -------------------------------------------------
\begin{frame}
  \titlepage
\end{frame}
% -------------------------------------------------
%%%%%%%%%%%%%%%%%%%%%%%%%%%%%%%%%%%%%%%%%%%%%%%%%%%%%%%%%%%%%%%%%%%%%%%%
%%%%%%%%%%%%%%%%%%%%%%%%%%%%%%%%%%%%%%%%%%%%%%%%%%%%%%%%%%%%%%%%%%%%%%%%
\begin{frame}
\frametitle{Why}
\begin{itemize}
  \item Central to the PageRank (and many many other applications in
fincance, science, informatics, etc) is that we randomly process
something
  \item what we want to know is ``on average'' what is likely to happen
  \item what would happen if we have an infinite number of samples?
  \item let's take a look at integral (a discrete limit in a sense)
\end{itemize}
\end{frame}
%%%%%%%%%%%%%%%%%%%%%%%%%%%%%%%%%%%%%%%%%%%%%%%%%%%%%%%%%%%%%%%%%%%%%%%%%
%%%%%%%%%%%%%%%%%%%%%%%%%%%%%%%%%%%%%%%%%%%%%%%%%%%%%%%%%%%%%%%%%%%%%%%%
\begin{frame}
\frametitle{Integration}
\framesubtitle{some slides from f. pellanccini}
  \begin{itemize}
  \item integral of a function over a domain
\begin{equation*} 
\int_{x\in D} f(x) \,dA_x
\end{equation*}
  \item the size of a domain
\begin{equation*} 
A_D = \int_{x\in D} dA_x
\end{equation*}
  \item average of a function over some domain
\begin{equation*} 
\frac{\int_{x\in D} f(x) dA_x}{A_D}
\end{equation*}
\end{itemize}
\end{frame}
%%%%%%%%%%%%%%%%%%%%%%%%%%%%%%%%%%%%%%%%%%%%%%%%%%%%%%%%%%%%%%%%%%%%%%%%%
%%%%%%%%%%%%%%%%%%%%%%%%%%%%%%%%%%%%%%%%%%%%%%%%%%%%%%%%%%%%%%%%%%%%%%%%
\begin{frame}
\frametitle{integral example}
The average ``daily'' snowfall in Champaign last year
  \begin{itemize}
    \item domain: year (1d time interval)
    \item integration variable: day
    \item function: snowfall depending on day
  \end{itemize}
\begin{equation*}
  average = \frac{\int_{day\in year} s(day)d_{day}}{length of year}
\end{equation*}
\end{frame}
%%%%%%%%%%%%%%%%%%%%%%%%%%%%%%%%%%%%%%%%%%%%%%%%%%%%%%%%%%%%%%%%%%%%%%%%%
%%%%%%%%%%%%%%%%%%%%%%%%%%%%%%%%%%%%%%%%%%%%%%%%%%%%%%%%%%%%%%%%%%%%%%%%
\begin{frame}
\frametitle{integral example}
The average snowfall in Illinois
  \begin{itemize}
    \item domain: Illinois (2d surface)
    \item integration variable: $(x,y)$ location
    \item function: snowfall depending on location
  \end{itemize}
\begin{equation*}
  average = \frac{\int_{location\in Illinois} s(location)d_{location}}{area of illinois}
\end{equation*}
\end{frame}
%%%%%%%%%%%%%%%%%%%%%%%%%%%%%%%%%%%%%%%%%%%%%%%%%%%%%%%%%%%%%%%%%%%%%%%%%
%%%%%%%%%%%%%%%%%%%%%%%%%%%%%%%%%%%%%%%%%%%%%%%%%%%%%%%%%%%%%%%%%%%%%%%%
\begin{frame}
\frametitle{integral example}
The average snowfall in Illinois today
  \begin{itemize}
    \item domain: Illinois $\times$ year (3d space-time)
    \item integration variable: location and day
    \item function: snowfall depending on location and day
  \end{itemize}
\begin{equation*}
  average = \frac{\int_{day\in year}\int_{location\in Illinois}s(location,day)d_{location,day}}{area of illinois \cdot length of year}
\end{equation*}
\end{frame}
%%%%%%%%%%%%%%%%%%%%%%%%%%%%%%%%%%%%%%%%%%%%%%%%%%%%%%%%%%%%%%%%%%%%%%%%%
%%%%%%%%%%%%%%%%%%%%%%%%%%%%%%%%%%%%%%%%%%%%%%%%%%%%%%%%%%%%%%%%%%%%%%%%
\begin{frame}
\frametitle{discrete random variables}
\begin{itemize}
  \item random variable $x$
  \item values: $x_0, x_1, \dots, x_n$
  \item probabilities $p_0, p_1,\dots,p_n$ with $\sum_{i=0}^{n}p_i = 1$
\end{itemize}
\begin{block}{throwing a die (1-based index)}
\begin{itemize}
  \item values: $x_1=1, x_2=2, \dots, x_6=6$
  \item probabilities $p_i = 1/6$
\end{itemize}
\end{block}  
\end{frame}
%%%%%%%%%%%%%%%%%%%%%%%%%%%%%%%%%%%%%%%%%%%%%%%%%%%%%%%%%%%%%%%%%%%%%%%%%
%%%%%%%%%%%%%%%%%%%%%%%%%%%%%%%%%%%%%%%%%%%%%%%%%%%%%%%%%%%%%%%%%%%%%%%%
\begin{frame}
\frametitle{expected value and variance}
\begin{itemize}
  \item expected value: average value of the variable
\begin{equation*}
  E[x] = \sum_{j=1}^{n}x_j p_j
\end{equation*}
  \item variance: variation from the average
\begin{equation*}
  \sigma^2[x] = E[(x-E[x])^2] = E[x^2] - E[x]^2
\end{equation*}
\end{itemize}
\begin{block}{throwing a die}
\begin{itemize}
  \item expected value: $E[x] = (1+2+\dots+6)/6 = 3.5$
  \item variance: $\sigma^2[x] = 2.916$
\end{itemize}
\end{block}  
\end{frame}
%%%%%%%%%%%%%%%%%%%%%%%%%%%%%%%%%%%%%%%%%%%%%%%%%%%%%%%%%%%%%%%%%%%%%%%%%
%%%%%%%%%%%%%%%%%%%%%%%%%%%%%%%%%%%%%%%%%%%%%%%%%%%%%%%%%%%%%%%%%%%%%%%%
\begin{frame}
\frametitle{estimated $E[x]$}
\begin{itemize}
  \item to estimate the expected value, choose a set of random values
based on the probability and average the results
\begin{equation*}
  E[x] = \frac{1}{N} \sum_{j=1}^{N} x_i
\end{equation*}
  \item bigger $N$ gives better estimates
\end{itemize}
\begin{block}{throwing a die}
\begin{itemize}
  \item 3 rolls: $3,1,6 \rightarrow E[x] \approx (3+1+6)/3 = 3.33$
  \item 9 rolls: $3,1,6,2,5,3,4,6,2 \rightarrow E[x] \approx (3+1+6+2+5+3+4+6+2)/9 = 3.51$
\end{itemize}
\end{block}  
\end{frame}
%%%%%%%%%%%%%%%%%%%%%%%%%%%%%%%%%%%%%%%%%%%%%%%%%%%%%%%%%%%%%%%%%%%%%%%%%
%%%%%%%%%%%%%%%%%%%%%%%%%%%%%%%%%%%%%%%%%%%%%%%%%%%%%%%%%%%%%%%%%%%%%%%%
\begin{frame}
\frametitle{law of large numbers}
\begin{itemize}
  \item by taking $N$ to $\infty$, the error between the estimate an the
expected value is statistically zero.  That is, the estimate will
converge to the correct value
\begin{equation*} 
  P\left(E[x] = lim_{N\rightarrow\infty} \frac{1}{N} \sum_{i=1}^{N} x_i\right) = 1
\end{equation*}
\end{itemize}
\end{frame}
%%%%%%%%%%%%%%%%%%%%%%%%%%%%%%%%%%%%%%%%%%%%%%%%%%%%%%%%%%%%%%%%%%%%%%%%%
%%%%%%%%%%%%%%%%%%%%%%%%%%%%%%%%%%%%%%%%%%%%%%%%%%%%%%%%%%%%%%%%%%%%%%%%
\begin{frame}
\frametitle{continuous random variable}
\begin{itemize}
  \item random variable: $x$
  \item values: $x\in [a,b]$
  \item probability: density function $\rho(x)$ with $\int_a^b
\rho(x)\,dx = 1$
  \item probability that the variable as value $x$: $\rho(x)$
\end{itemize}
\end{frame}
%%%%%%%%%%%%%%%%%%%%%%%%%%%%%%%%%%%%%%%%%%%%%%%%%%%%%%%%%%%%%%%%%%%%%%%%%
%%%%%%%%%%%%%%%%%%%%%%%%%%%%%%%%%%%%%%%%%%%%%%%%%%%%%%%%%%%%%%%%%%%%%%%%
\begin{frame}
\frametitle{uniformly distributed random variable}
\begin{itemize}
  \item $\rho(x)$ is constant
  \item $\int_a^b \rho(x)\,dx = 1$ means $\rho(x) = 1/(b-a)$
\end{itemize}
\end{frame}
%%%%%%%%%%%%%%%%%%%%%%%%%%%%%%%%%%%%%%%%%%%%%%%%%%%%%%%%%%%%%%%%%%%%%%%%%
%%%%%%%%%%%%%%%%%%%%%%%%%%%%%%%%%%%%%%%%%%%%%%%%%%%%%%%%%%%%%%%%%%%%%%%%
\begin{frame}
\frametitle{continuous extensions}
\begin{itemize}
  \item expected value
  \begin{align*}
  E[x] & = \int_a^b x\rho(x)\,dx\\
  E[g(x)] & = \int_a^b g(x)\rho(x)\,dx
  \end{align*}
  \item variance
  \begin{align*}
  \sigma^2[x] & = \int_a^b (x-E[x])^2\rho(x)\,dx\\
  \sigma^2[g(x)] & = \int_a^b (g(x)-E[g(x)])^2 p(x)\,dx\\
  \end{align*}
  \item estimating the expected value
  \begin{equation*}
    E[g(x)] \approx \frac{1}{N} \sum_{i=1}^{N} g(x_i)
\end{equation*}
\end{itemize}
\end{frame}
%%%%%%%%%%%%%%%%%%%%%%%%%%%%%%%%%%%%%%%%%%%%%%%%%%%%%%%%%%%%%%%%%%%%%%%%%
%%%%%%%%%%%%%%%%%%%%%%%%%%%%%%%%%%%%%%%%%%%%%%%%%%%%%%%%%%%%%%%%%%%%%%%%
\begin{frame}
\frametitle{multidementional extensions}
\begin{itemize}
  \item difficult domains (complex geometries)
  \item expected value
  \begin{align*}
  E[g(x)] & = \int_{x\in D} g(x)\rho(x)\,dA_x
  \end{align*}
\end{itemize}
\end{frame}
%%%%%%%%%%%%%%%%%%%%%%%%%%%%%%%%%%%%%%%%%%%%%%%%%%%%%%%%%%%%%%%%%%%%%%%%%
%%%%%%%%%%%%%%%%%%%%%%%%%%%%%%%%%%%%%%%%%%%%%%%%%%%%%%%%%%%%%%%%%%%%%%%%
\begin{frame}
\frametitle{(deterministic) numerical integration}
\begin{itemize}
  \item  split domain into set of fixed segments
  \item sum function values with size of segments (Riemann!)
\end{itemize}
\begin{center}
  \pgfimage[height=4cm]{./figs/map2}
\end{center}
\end{frame}
%%%%%%%%%%%%%%%%%%%%%%%%%%%%%%%%%%%%%%%%%%%%%%%%%%%%%%%%%%%%%%%%%%%%%%%%%
%%%%%%%%%%%%%%%%%%%%%%%%%%%%%%%%%%%%%%%%%%%%%%%%%%%%%%%%%%%%%%%%%%%%%%%%
\begin{frame}[fragile]
\frametitle{Algorithm}
We have for a random sequence $x_1,\dots,x_n$
\begin{equation*}
  \int_0^1 f(x)\,dx \approx \frac{1}{n} \sum_{i=1}^{n} f(x_i)  
\end{equation*}

\begin{lstlisting}[mathescape]
  n=100;
  x=rand(n,1);
  a=f(x);
  s=sum(a)/n;
\end{lstlisting}
\end{frame}
%%%%%%%%%%%%%%%%%%%%%%%%%%%%%%%%%%%%%%%%%%%%%%%%%%%%%%%%%%%%%%%%%%%%%%%%%
%%%%%%%%%%%%%%%%%%%%%%%%%%%%%%%%%%%%%%%%%%%%%%%%%%%%%%%%%%%%%%%%%%%%%%%%
\begin{frame}
\frametitle{2d: example computing $\pi$}
Use the unit square $[0,1]^2$ with a quarter-circle
\begin{align*}
  f(x,y) & = 
\begin{cases}
1 & \quad (x,y)\in \, circle\\
0 & else
\end{cases}\\
  A_{quarter-circle} &= \int_0^1\int_0^1 f(x,y)\,dxdy = \frac{pi}{4}
\end{align*}
\begin{center}
  \pgfimage[height=4cm,width=4cm]{./figs/map4}
\end{center}
\end{frame}
%%%%%%%%%%%%%%%%%%%%%%%%%%%%%%%%%%%%%%%%%%%%%%%%%%%%%%%%%%%%%%%%%%%%%%%%%
%%%%%%%%%%%%%%%%%%%%%%%%%%%%%%%%%%%%%%%%%%%%%%%%%%%%%%%%%%%%%%%%%%%%%%%%
\begin{frame}
\frametitle{2d: example computing $\pi$}
Estimate the area of the circle by randomly evaluating $f(x,y)$
\begin{align*}
  A_{quarter-circle} &\approx \frac{1}{N}\sum_{i=1}^{N}f(x_i,y_i)
\end{align*}
\begin{center}
  \pgfimage[height=4cm,width=4cm]{./figs/map3}
\end{center}
\end{frame}
%%%%%%%%%%%%%%%%%%%%%%%%%%%%%%%%%%%%%%%%%%%%%%%%%%%%%%%%%%%%%%%%%%%%%%%%%
%%%%%%%%%%%%%%%%%%%%%%%%%%%%%%%%%%%%%%%%%%%%%%%%%%%%%%%%%%%%%%%%%%%%%%%%
\begin{frame}
\frametitle{2d: example computing $\pi$}
By definition
\begin{align*}
  A_{quarter-circle} & = \pi/4
\end{align*}
so
\begin{equation*}
  \pi \approx \frac{4}{N} \sum_{i=1}^{N}f(x_i,y_i)
\end{equation*}
\end{frame}
%%%%%%%%%%%%%%%%%%%%%%%%%%%%%%%%%%%%%%%%%%%%%%%%%%%%%%%%%%%%%%%%%%%%%%%%%
%%%%%%%%%%%%%%%%%%%%%%%%%%%%%%%%%%%%%%%%%%%%%%%%%%%%%%%%%%%%%%%%%%%%%%%%
\begin{frame}[fragile]
\frametitle{2d: example computing $\pi$, algorithm}
\begin{lstlisting}[mathescape]
  input $N$
  call rand in $2d$
  for $i$=1:$N$
    $sum = sum + f(x_i,y_i)$
  end
  $sum = 4*sum/N$
\end{lstlisting}
\end{frame}
%%%%%%%%%%%%%%%%%%%%%%%%%%%%%%%%%%%%%%%%%%%%%%%%%%%%%%%%%%%%%%%%%%%%%%%%%
%%%%%%%%%%%%%%%%%%%%%%%%%%%%%%%%%%%%%%%%%%%%%%%%%%%%%%%%%%%%%%%%%%%%%%%%
\begin{frame}
\frametitle{2d: example computing $\pi$, algorithm}
The expected value of the \emph{error} is $\mO\left(\frac{1}{\sqrt{N}}\right)$
\begin{itemize}
  \item convergence does not depend on dimension
  \item deterministic integration is very hard in higher dimensions
  \item deterministic integration is very hard for complicated domains
\end{itemize}
\end{frame}
%%%%%%%%%%%%%%%%%%%%%%%%%%%%%%%%%%%%%%%%%%%%%%%%%%%%%%%%%%%%%%%%%%%%%%%%%
%%%%%%%%%%%%%%%%%%%%%%%%%%%%%%%%%%%%%%%%%%%%%%%%%%%%%%%%%%%%%%%%%%%%%%%%
\begin{frame}
\frametitle{CLT: Central Limit Theorem}
\begin{block}{}
  The distribution of an average is close to being normal, even when the distribution from which the average is computed is not normal.
\end{block}
What?
\begin{itemize}
  \item Let $x_1,\dots,x_n$ be some independent random variables from any PDF
  \item Consider the sum $S_n = x_1 + \dots + x_n$
  \item The expected value is $n\mu$ and the standard deviation is
$\sigma \sqrt{n}$
  \item That is, $\frac{S_n - n\mu}{\sigma\sqrt{n}}$ approaches the
normal distribution
  \item What?  The sample mean has an error of $\sigma/\sqrt{n}$
\end{itemize}

\end{frame}
%%%%%%%%%%%%%%%%%%%%%%%%%%%%%%%%%%%%%%%%%%%%%%%%%%%%%%%%%%%%%%%%%%%%%%%%%
%%%%%%%%%%%%%%%%%%%%%%%%%%%%%%%%%%%%%%%%%%%%%%%%%%%%%%%%%%%%%%%%%%%%%%%%
\begin{frame}
\frametitle{Now What?}
\begin{itemize}
  \item How does one minimize the noise in this random process?
  \item pick better samples!
  \item Use more samples where the impact is high: where $f$ is large
  \begin{equation*}
  I \approx \frac{1}{N} \sum_{i=1}^{N} \frac{f(x)}{p(x)}
\end{equation*}
  \item So pick a distribution similar to $f$
  \begin{equation*}
  p_{optimal} \propto f(x)
\end{equation*}
\end{itemize}
\end{frame}
%%%%%%%%%%%%%%%%%%%%%%%%%%%%%%%%%%%%%%%%%%%%%%%%%%%%%%%%%%%%%%%%%%%%%%%%%
\begin{frame}[fragile]
\frametitle{\texttt{monte\_pi.c} vs. \texttt{monte\_pi.m}}
\begin{lstlisting}[caption=C]
... estimate of pi is 3.14155
\end{lstlisting}
\begin{lstlisting}[caption=matlab]
... estimate of pi is 3.141597
\end{lstlisting}

\begin{alertblock}{}
Be careful with your random number generator.
\end{alertblock}
\end{frame}
\end{document}

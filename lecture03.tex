% Copyright Luke Olson 2009--2014
% This work is licensed under the Creative Commons
% Attribution-NonCommercial-NoDerivatives 4.0 International License. To view a
% copy of this license, visit http://creativecommons.org/licenses/by-nc-nd/4.0/.
%
\documentclass[10pt]{beamer}
%\documentclass[handout,10pt]{beamer}
%
\mode<presentation>
{
  \usetheme[secheader]{Boadilla}
  \usecolortheme{luke}
  \usefonttheme[onlymath]{serif}
  \setbeamercovered{invisible}
  %\setbeamercovered{transparent}
  %
}
\mode<handout>
{
  \usetheme[secheader]{Boadilla}
  \usecolortheme{luke2}
  \usefonttheme[onlymath]{serif}
  \setbeamercovered{invisible}
  %\setbeamercovered{transparent}
}

\usepackage{pgf,pgfarrows,pgfnodes,pgfautomata,pgfheaps,pgfshade}
\usepackage{pxfonts}
\usepackage{eulervm}
\usepackage{listings}
%
%
%
\newcommand{\vb}{{\bf{b}}}
\newcommand{\ve}{{\bf{e}}}
\newcommand{\vg}{{\bf{g}}}
\newcommand{\vp}{{\bf{p}}}
\newcommand{\vr}{{\bf{r}}}
\newcommand{\vu}{{\bf{u}}}
\newcommand{\vx}{{\bf{x}}}
\newcommand{\vz}{{\bf{z}}}
\newcommand{\vA}{{\bf{A}}}
\newcommand{\vU}{{\bf{U}}}
\newcommand{\mO}{{\mathcal{O}}}
\newcommand{\mF}{{\mathcal{F}}}
\definecolor{mygray}{rgb}{0.95,0.95,0.95}
\lstset{
        language=matlab,
        numbers=left, numberstyle=\tiny, stepnumber=1, numbersep=5pt,
        basicstyle=\color{black}\ttfamily\small,
        commentstyle=\color{green}\ttfamily,
        keywordstyle=\color{blue}\ttfamily,
        stringstyle=\color{red}\ttfamily,
        showstringspaces=false,
        backgroundcolor=\color{mygray},
        breaklines,
}

\author{L. Olson}
\institute[UIUC]
{Department of Computer Science\\
University of Illinois at Urbana-Champaign\\
\vspace{0.5cm}
}
%%%%%%%%%%%%%%%%%%%%%%%%%%%%%%%%%%%%%%%%%%%%%%%%%%%%%%%%%%%%%%%%%%%%%%%%
%%%%%%%%%%%%%%%%%%%%%%%%%%%%%%%%%%%%%%%%%%%%%%%%%%%%%%%%%%%%%%%%%%%%%%%%
\title[CS 357]{Lecture 3}
\subtitle{Numerical Representation}
\date{September 1, 2009}

\begin{document}
% -------------------------------------------------
\begin{frame}
  \titlepage
\end{frame}
% -------------------------------------------------
% -------------------------------------------------
\begin{frame}
\frametitle{What we'll do:}
\begin{itemize}
  \item look at floating point representation in its basic form
  \item expose errors of a different form: rounding error
  \item highlight IEEE-754 standard
\end{itemize}
\end{frame}
% -------------------------------------------------
% -------------------------------------------------
\begin{frame}
\frametitle{Why this is important:}
\begin{itemize} 
  \item Errors come in two forms: truncation error and rounding error
    \begin{itemize}
      \item we always have them...
      \item case study: Intel
      \item our jobs as developers: reduce impact
    \end{itemize}
\end{itemize}
\end{frame}
% -------------------------------------------------
\begin{frame}
\frametitle{Next: Floating Point Numbers}
  \begin{itemize}
    \item We're familiar with base 10 representation of numbers:
      \begin{equation*}
        1234 = 4\times 10^0 + 3\times 10^1 + 2\times 10^2 + 1\times 10^3
      \end{equation*}
      and
      \begin{equation*}
        .1234 = 1\times 10^{-1} + 2\times 10^{-2} + 3\times 10^{-3} +
4\times 10^{-4}
\end{equation*}
    \item we write 1234.1234 as an integer part and a fractional part:
      \begin{equation*}
        a_3a_2a_1a_0.b_1b_2b_3b_4
\end{equation*}
    \item For some (even simple) numbers, there may be an
\emph{infinite} number of digits to the right:
      \begin{align*}  
          \pi & = 3.14159\dots\\
          1/9 & = 0.11111\dots\\
          \sqrt{2} & = 1.41421\dots\\  
      \end{align*}
\end{itemize}
\end{frame}
% -------------------------------------------------
% -------------------------------------------------
\begin{frame}
\frametitle{Other bases}
  \begin{itemize}
    \item So far, we have just base 10.  What about base $\beta$?
    \item binary ($\beta = 2$), octal ($\beta = 8$), hexadecimal ($\beta
= 16$), etc
    \item In the $\beta$-system we have
  \begin{equation*}
        (a_n\dots a_2a_1a_0.b_1b_2b_3b_4\dots)_{\beta} =
\sum_{k=0}^{n}a_k\beta^{k} + \sum_{k=0}^{\infty} b_k \beta^{-k}
\end{equation*}
\end{itemize}
\end{frame}
% -------------------------------------------------
% -------------------------------------------------
\begin{frame}[shrink]
\frametitle{Integer Conversion}
An algorithm to compute the base 2 representation of a base 10 integer
\begin{align*}
(N)_{10} & = (a_j a_{j-1}\dots a_2 a_0)_2\\
& = a_j\cdot 2^j + \dots + a_1\cdot 2^1 + a_0\cdot 2^0
\end{align*}
Compute $(N)_{10}/2 = Q + R$:
\begin{equation*}
  \frac{N}{2} = 
\underbrace{a_j\cdot 2^{j-1} + \dots + a_1\cdot 2^0}_{=Q} +
\underbrace{\frac{a_0}{2}}_{=R/2}
\end{equation*}
\begin{example}{Example: compute $(11)_{10}$ base 2}
\begin{align*}
11/2 = 5R1 \quad\Rightarrow\quad a_0 = 1\\
5/2 = 2R1 \quad\Rightarrow\quad a_1 = 1\\
2/2 = 1R0 \quad\Rightarrow\quad a_2 = 0\\
1/2 = 0R1 \quad\Rightarrow\quad a_3 = 1\\
\end{align*}
So $(11)_{10} = (1011)_{2}$
\end{example}
\end{frame}
% -------------------------------------------------
% -------------------------------------------------
\begin{frame}[fragile]
\frametitle{The other way...}
  Convert a base-2 number to base-10:
\begin{align*}
  &(11\,000\,101)_2 \\
& = 
1\times 2^7 +
1\times 2^6 +
0\times 2^5 +
0\times 2^4 +
0\times 2^3 +
1\times 2^2 +
0\times 2^1 +
1\times 2^0\\
& = 1 + 2(0 + 2(1 + 2(0 + 2(0 + 2(0 + 2(1 + 2(1)))))))\\
& = 197
\end{align*}

\begin{block}{Matlab}
\begin{lstlisting}
>> dec2bin(197)
ans = 11000101

>> bin2dec('11000101')
ans = 197
\end{lstlisting}
\end{block}
\end{frame}
% -------------------------------------------------
% -------------------------------------------------
\begin{frame}
\frametitle{Converting fractions}
\begin{itemize}
  \item straight forward way is not easy
  \item goal: for $x\in[0,1]$ write
  \begin{equation*}
    x = 0.b_1b_2b_3b_4\dots = \sum_{k=1}^{\infty} c_k \beta^{-k} =
(0.c_1c_2c_3\dots )_{\beta}
\end{equation*}
  \item $\beta(x) = (c_1.c_2c_3c_4\dots)_{\beta}$
  \item multiplication by $\beta$ in base-$\beta$ only shifts the radix
\end{itemize}
\end{frame}
% -------------------------------------------------
% -------------------------------------------------
\begin{frame}[shrink]
\frametitle{Fraction Algorithm}
An algorithm to compute the binary representation of a fraction $x$:
\begin{align*}
x &= 0.b_1b_2b_3b_4\dots\\
& = b_1\cdot 2^{-1} + \dots\\
\end{align*}
Multiply $x$ by 2.  The integer part of $2x$ is $b_1$
\begin{equation*}
2x = b_1\cdot 2^0 + b_2\cdot 2^{-1} + b_3\cdot 2^{-2} + \dots
\end{equation*}
\begin{example}{Example:Compute the binary representation of 0.625}
\begin{align*}
2\cdot 0.625 = 1.25 \quad\Rightarrow\quad b_{-1} = 1\\
2\cdot 0.25 = 0.5 \quad\Rightarrow\quad b_{-2} = 0\\
2\cdot 0.5 = 1.0 \quad\Rightarrow\quad b_{-3} = 1\\
\end{align*}
So $(0.625)_{10} = (0.101)_{2}$
\end{example}
\end{frame}
% -------------------------------------------------
% -------------------------------------------------
\begin{frame}[fragile]
\frametitle{A problem with precision}
\begin{lstlisting}[mathescape]
$r_0 = x$
for $k=1,2,\dots,m$
  if $r_{k-1} \geq 2^{-k}$
    $b_k = 1$
    $r_k = r_{k-1} - 2^{-k}$
  else
    $b_k = 0$
  end 
end
\end{lstlisting}
\begin{tabular}{l | l | l | l}\hline
$k$ & $2^{-k}$ & $b_k$ & $r_k = r_{k-1} - b_{k}2^{-k}$\\\hline
0 & & & 0.8125 \\
1 & 0.5    & 1 & 0.3125 \\
2 & 0.25   & 1 & 0.0625 \\
3 & 0.125  & 0 & 0.0625 \\
4 & 0.0625 & 1 & 0.0000 \\
\end{tabular}
\end{frame}
% -------------------------------------------------
% -------------------------------------------------
\begin{frame}
\frametitle{A problem with precision}
For other numbers, such as $\frac{1}{5} = 0.2$, an infinite length
is needed.

\begin{block}{}
$0.2 \quad\rightarrow\quad .0011\, 0011\, 0011\, \dots$

\vspace{0.5cm}
So 0.2 is stored just fine in base-10, but needs infinite number of digits
in base-2
\end{block}

\begin{alertblock}{!!!}
This is \emph{roundoff} error in its basic form...
\end{alertblock}
\end{frame}
% -------------------------------------------------
% -------------------------------------------------
\begin{frame}[shrink]
\frametitle{Intel}
\pgfimage[width=\textwidth]{figs/intel1}
\end{frame}
% -------------------------------------------------
% -------------------------------------------------
\begin{frame}[shrink]
\frametitle{Intel}
\pgfimage[width=\textwidth]{figs/intel2}
\end{frame}
% -------------------------------------------------
% -------------------------------------------------
\begin{frame}[shrink]
\frametitle{Intel}
\pgfimage[width=\textwidth]{figs/intel3}
\end{frame}
% -------------------------------------------------
% -------------------------------------------------
\begin{frame}
\frametitle{Intel Timeline}
\framesubtitle{from emery.com}
\begin{description}
\item[June 1994]  Intel engineers discover the division error.  Managers
decide the error will not impact many people.  Keep the issue internal.
\item[June 1994] Dr Nicely at Lynchburg College notices computation problems
\item[Oct 19, 1994] After months of testing, Nicely confirms that other
errors are not the cause.  The problem is in the Intel Processor.
\item[Oct 24, 1994] Nicely contacts Intel.  Intel duplicates error.
\item[Oct 30, 1994] After no action from Intel, Nicely sends an email
\end{description}
\end{frame}
% -------------------------------------------------
% -------------------------------------------------
\begin{frame}[containsverbatim,shrink]
\frametitle{Intel Timeline}
\framesubtitle{from emery.com}
\begin{block}{}
\begin{verbatim}
FROM:  Dr. Thomas R. Nicely
       Professor of Mathematics
       Lynchburg College
       1501 Lakeside Drive
       Lynchburg, Virginia 24501-3199

       Phone:     804-522-8374
       Fax:       804-522-8499
       Internet:  nicely@acavax.lynchburg.edu

TO:    Whom it may concern

RE:    Bug in the Pentium FPU

DATE:  30 October 1994

It appears that there is a bug in the floating point unit
(numeric coprocessor) of many, and perhaps all, Pentium
processors.

In short, the Pentium FPU is returning erroneous values 
for certain division operations. For example,
     0001/824633702441.0
is calculated incorrectly (all digits beyond the eighth
significant digit are in error). This can be verified in
compiled code, an ordinary spreadsheet such as Quattro Pro
or Excel, or even the Windows calculator (use the
scientific mode), by computing
     00(824633702441.0)*(1/824633702441.0),
which should equal 1 exactly (within some extremely small
rounding error; in general, coprocessor results should 
contain 19 significant decimal digits).  However, the
Pentiums tested return
     0000.999999996274709702
.
.
.
\end{verbatim}
\end{block}
\end{frame}
% -------------------------------------------------
% -------------------------------------------------
\begin{frame}
\frametitle{Intel Timeline}
\framesubtitle{from emery.com}
\begin{description}
\item[Nov 1, 1994]  Software company Phar Lap Software receives Nicely's
email.  Sends to colleagues at Microsoft, Borland, Watcom, etc.
decide the error will not impact many people.  Keep the issue internal.
\item[Nov 2, 1994]  Email with description goes global.
\item[Nov 15, 1994] USC reverse-engineers the chip to expose the problem.
Intel still denies a problem.  Stock falls.
\item[Nov 22, 1994] CNN \emph{Moneyline} interviews Intel.  Says the problem
is minor.
\item[Nov 23, 1994] The MathWorks develops a fix.
\item[Nov 24, 1994] New York Times story.  Intel still sending out flawed
chips.  Will replace chips only if it caused a problem in an important
application.
\item[Dec 12, 1994] IBM halts shipment of Pentium based PCs
\item[Dec 16, 1994] Intel stock falls again.
\item[Dec 19, 1994] More reports in the NYT: lawsuits, etc.
\item[Dec 20, 1994] Intel admits.  Sets aside \$420 million to fix.
\end{description}
\end{frame}
% -------------------------------------------------
% -------------------------------------------------
\begin{frame}
\frametitle{Numerical "bugs"}
\begin{alertblock}{Obvious}
Software has bugs
\end{alertblock}

\vspace{1.0cm}
\begin{alertblock}{Not-SO-Obvious}
Numerical software has two unique bugs:
\begin{enumerate}
  \item roundoff error
  \item truncation error
\end{enumerate}
\end{alertblock}
\end{frame}
% -------------------------------------------------
% -------------------------------------------------
\begin{frame}
\frametitle{Numerical Errors}
\begin{block}{Roundoff}
\emph{Roundoff} occurs when digits in a decimal point (0.3333...) are lost
(0.3333) due to a limit on the memory available for storing one numerical
value.
\end{block}

\vspace{1.0cm}
\begin{block}{Truncation}
\emph{Truncation} error occurs when discrete values are used to
approximate a mathematical expression.
\end{block}
\end{frame}
% -------------------------------------------------
% -------------------------------------------------
\begin{frame}
\frametitle{Uncertainty: well- or ill-conditioned?}
\vspace{1.0cm}
Errors in input data can cause \emph{uncertain} results
\begin{itemize}
  \item input data can be experimental or rounded.  leads to a certain
  variation in the results
  \item \emph{well-conditioned}: numerical results are insensitive to small
  variations in the input
  \item \emph{ill-conditioned}: small variations lead to drastically different
  numerical calculations (a.k.a. poorly conditioned)
\end{itemize}
\end{frame}
% -------------------------------------------------
% -------------------------------------------------
\begin{frame}
\frametitle{Our Job}
As numerical analysts, we need to
\begin{enumerate}
  \item solve a problem so that the calculation is not susceptible to large
  roundoff error
  \item solve a problem so that the approximation has a \emph{tolerable}
  truncation error
\end{enumerate}

\vspace{1.0cm}
How?
\begin{itemize}
  \item incorporate roundoff-truncation knowledge into
  \begin{itemize}
    \item the mathematical model
    \item the method
    \item the algorithm
    \item the software design
  \end{itemize}
  \item awareness $\rightarrow$ correct interpretation of results
\end{itemize}
\end{frame}
\begin{frame}
\frametitle{Floating Points}
\begin{block}{Normalized Floating-Point Representation}
  Real numbers are stored as
  \begin{equation*}
    x = \pm(0.d_1d_2d_3\dots d_m)_{\beta} \times \beta^{e}
\end{equation*}
\end{block}
\begin{itemize}
  \item $d_1d_2d_3\dots d_m$ is the mantissa, $e$ is the exponent
  \item $e$ is negative, positive or zero
  \item the general normalized form requires $d_1 \ne 0$
\end{itemize}
\end{frame}
% -------------------------------------------------
% -------------------------------------------------
\begin{frame}
\frametitle{Floating Point}
\begin{example}
  In base 10
\begin{itemize}
  \item 1000.12345 can be written as
\begin{equation*}
  (0.100012345)_{10} \times 10^4
\end{equation*}
  \item 0.000812345 can be written as
\begin{equation*}
  (0.812345)_{10} \times 10^{-3}
\end{equation*}
\end{itemize}
\end{example}
\end{frame}
% -------------------------------------------------
% -------------------------------------------------
\begin{frame}
\frametitle{Floating Point}
  Suppose we have only 3 bits for a mantissa and a 1 bit exponent stored
like

\begin{center}
\begin{tabular}{|c|c|c||c|}\hline
  $.d_1$ & $d_2$ & $d_3$ & $e_1$\\\hline
\end{tabular}
\end{center}

All possible combinations give:
\begin{align*}
000_2 & = 0\\
\dots & \qquad\qquad \times 2^{-1,0,1}\\
111_2 & = 7\\
\end{align*}
So we get $0,\frac{1}{16},\frac{2}{16},\dots,\frac{7}{16}$, 
$0,\frac{1}{4},\frac{2}{4},\dots,\frac{7}{4}$, 
and $0,\frac{1}{8},\frac{2}{8},\dots,\frac{7}{8}$.  On the real line:

\begin{center}
  \pgfimage[width=12cm]{figs/fp4bit}
\end{center}

\end{frame}
% -------------------------------------------------
% -------------------------------------------------
\begin{frame}
\frametitle{Overflow, Underflow}
\begin{center}
  \pgfimage[width=12cm]{figs/fp4bit}
\end{center}
\begin{itemize}
  \item computations too close to zero may result in \emph{underflow}
  \item computations too large may result in \emph{overflow}
  \item overflow error is considered more severe
  \item underflow can just fall back to 0
\end{itemize}
\end{frame}
% -------------------------------------------------
% -------------------------------------------------
\begin{frame}
\frametitle{Normalizing}
If we use the normalized form in our 4-bit case, we lose $0.001_2\times
2^{-1,0,1}$ along with other.  So we cannot represent
$\frac{1}{16}$,
$\frac{1}{8}$,
and $\frac{3}{16}$.

\begin{center}
  \pgfimage[width=12cm]{figs/fp4bit}
\end{center}
\begin{center}
  \pgfimage[width=12cm]{figs/fp4bitnormalized}
\end{center}

\end{frame}
% -------------------------------------------------
% -------------------------------------------------
\begin{frame}
\frametitle{}
More on numerical representation...
\begin{itemize}
  \item read 
    \begin{description}
      \item[5th ed.] end of section \S 2.1 in NMC
      \item[6th ed.] appendix B.1 in NMC
    \end{description}
\end{itemize}

\bigskip
Next:
\begin{itemize}
  \item Floating Points
  \begin{itemize}
    \item IEEE 754
    \item mantissa, exponent, machine epsilon, hidden bits
    \item roundoff, rounding
  \end{itemize}
  \item loss of significance
  \item precision
\end{itemize}
\end{frame}
% -------------------------------------------------
% -------------------------------------------------
\end{document}

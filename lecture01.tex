% Copyright Luke Olson 2009--2014
% This work is licensed under the Creative Commons
% Attribution-NonCommercial-NoDerivatives 4.0 International License. To view a
% copy of this license, visit http://creativecommons.org/licenses/by-nc-nd/4.0/.
%
\documentclass[10pt]{beamer}
%\documentclass[handout,10pt]{beamer}
%
\mode<presentation>
{
  \usetheme{Boadilla}
  \usecolortheme{luke}
  \usefonttheme[onlymath]{serif}
  \setbeamercovered{invisible}
  %\setbeamercovered{transparent}
}
\mode<handout>
{
  \usetheme{Boadilla}
  \usecolortheme{luke2}
  \usefonttheme[onlymath]{serif}
  \setbeamercovered{invisible}
  %\setbeamercovered{transparent}
}
\usepackage{pgf}
\usepackage{pxfonts}
\usepackage{eulervm}
\usepackage{listings}
%
%
%
\newcommand{\vb}{{\bf{b}}}
\newcommand{\ve}{{\bf{e}}}
\newcommand{\vg}{{\bf{g}}}
\newcommand{\vp}{{\bf{p}}}
\newcommand{\vr}{{\bf{r}}}
\newcommand{\vu}{{\bf{u}}}
\newcommand{\vx}{{\bf{x}}}
\newcommand{\vz}{{\bf{z}}}
\newcommand{\vA}{{\bf{A}}}
\newcommand{\vU}{{\bf{U}}}
\newcommand{\mO}{{\mathcal{O}}}
\newcommand{\mF}{{\mathcal{F}}}
\definecolor{mygray}{rgb}{0.95,0.95,0.95}
\lstset{
        language=matlab,
        numbers=left, numberstyle=\tiny, stepnumber=1, numbersep=5pt,
        basicstyle=\color{black}\ttfamily\small,
        commentstyle=\color{green}\ttfamily,
        keywordstyle=\color{blue}\ttfamily,
        stringstyle=\color{red}\ttfamily,
        showstringspaces=false,
        backgroundcolor=\color{mygray},
        breaklines,
}

\author{L. Olson}
\institute[UIUC]
{Department of Computer Science\\
University of Illinois at Urbana-Champaign\\
\vspace{0.5cm}
}
%%%%%%%%%%%%%%%%%%%%%%%%%%%%%%%%%%%%%%%%%%%%%%%%%%%%%%%%%%%%%%%%%%%%%%%%
%%%%%%%%%%%%%%%%%%%%%%%%%%%%%%%%%%%%%%%%%%%%%%%%%%%%%%%%%%%%%%%%%%%%%%%%
\title[CS 357]{Lecture 1}
\subtitle{Introduction to Numerical Methods}
\date{August 25, 2009}

\begin{document}
% -------------------------------------------------
\begin{frame}
  \titlepage
\end{frame}
% -------------------------------------------------
\begin{frame}
\begin{block}{}
\centering
Numerical Methods?\\

\vspace{1cm}

Numerical Analysis?
\end{block}
\end{frame}
% -------------------------------------------------
\begin{frame}
\frametitle{Numerical Calculation vs.~Symbolic Calculation}
\begin{itemize}
\item Numerical Calculation: involve numbers directly
\begin{itemize}
\item manipulate numbers to produce a {\bf numerical} result
\end{itemize}
\item Symbolic Calculation: symbols represent numbers 
\begin{itemize}
\item manipulate symbols according to mathematical rules to produce a
symbolic result
\end{itemize}
\end{itemize}
\begin{columns}
\begin{column}{0.45\textwidth}
\begin{example}[numerical]
\[
\frac{(17.36)^2 - 1}{17.36 + 1} = 16.36
\]
\end{example}
\end{column}
\begin{column}{0.45\textwidth}
\begin{example}[symbolic]
\[
\frac{x^2 - 1}{x + 1} = x-1
\]
\end{example}
\end{column}
\end{columns}
\end{frame}
% -------------------------------------------------
% -------------------------------------------------
\begin{frame}
\frametitle{Analytic Solution vs.~Numerical Solution}
\begin{itemize}
\item Analytic Solution (a.k.a. symbolic): The exact numerical or symbolic representation of
the solution
  \begin{itemize}
  \item may use special characters such as $\pi$, $e$, or $\tan{(83)}$
  \end{itemize}
\item Numerical Solution: The computational representation of the solution
  \begin{itemize}
  \item entirely numerical
  \end{itemize}
\end{itemize}
\begin{columns}
\begin{column}{0.45\textwidth}
\begin{example}[analytic]
\[
\frac{1}{4}
\]
\[
\frac{1}{3}
\]
\[
\pi
\]
\[
\tan{(83})
\]
\end{example}
\end{column}
\begin{column}{0.45\textwidth}
\begin{example}[numerical]
\[
0.25
\]
\[
0.33333\dots(?)
\]
\[
3.14159\dots(?)
\]
\[
0.88472\dots(?)
\]
\end{example}
\end{column}
\end{columns}
\end{frame}
% -------------------------------------------------
\begin{frame}
\frametitle{Numerical Computation and Approximation}
\begin{itemize}
\item Numerical Approximation is needed to carry out the steps in the
numerical calculation.  The overall process is a numerical computation.
\begin{example}[symbolic computation, numerical solution]
\[
\frac{1}{2} +
\frac{1}{3} +
\frac{1}{4} - 1 = \frac{1}{12} = 0.083333333\dots
\]
\end{example}
\begin{example}[numerical computation, numerical approximation]
\[
0.500 +
0.333 +
0.250
- 1.000 = 0.083
\]
\end{example}
\end{itemize}
\end{frame}
% -------------------------------------------------
\begin{frame}
\frametitle{Method vs.~Algorithm vs.~Implementation}
\begin{itemize}
\item Method: a general (mathematical) framework describing the solution process
\item Algorithm: a detailed description of executing the method
\item Implementation: a particular instantiation of the algorithm
\end{itemize}
\begin{itemize}
\item Is it a ``good'' method?
\item Is it a robust algorithm?
\item Is it a fast implementation?
\end{itemize}
\end{frame}
% -------------------------------------------------
% -------------------------------------------------
\begin{frame}
\frametitle{The Big Theme}
\begin{columns}
\begin{column}{0.45\textwidth}
\pgfimage[height=4cm]{./figs/pull1}

\begin{center}
  \begin{block}{}
    Accuracy
  \end{block}
\end{center}
\end{column}

\begin{column}{0.45\textwidth}
\pgfimage[height=4cm]{./figs/pull2}

\begin{center}
  \begin{block}{}
    Cost
  \end{block}
\end{center}
\end{column}

\end{columns}
\end{frame}
% -------------------------------------------------
% -------------------------------------------------
\begin{frame}
\frametitle{History: Numerical Algorithms}
\begin{itemize}
\item date to 1650 BC: The Rhind Papyrus of ancient Egypt contains 85
problems; many use numerical algorithms (T. Chartier, Davidson)
\end{itemize}
\begin{columns}
  \begin{column}{0.55\textwidth}
    \pgfimage[height=6cm]{./figs/rhind}
  \end{column}
  \begin{column}{0.35\textwidth}
    \begin{itemize}
        \item Approximates $\pi$ with $(8/9)^2 * 4 \approx 3.1605$
    \end{itemize}
  \end{column}
\end{columns}
\end{frame}
% -------------------------------------------------
% -------------------------------------------------
\begin{frame}
\frametitle{History: Archimedes}
\begin{columns}
  \begin{column}{0.55\textwidth}
    \begin{itemize}
    \item 287-212BC developed the "Method of Exhaustion"
    \item Method for determining $\pi$
      \begin{itemize}
        \item find the length of the permieter of a polygon inscribed inside a
        circle of radius $1/2$
        \item find the permiter of a polygon circumscribed outside a circle
        of radius $1/2$
        \item the value of $\pi$ is between these two lengths
      \end{itemize}
    \end{itemize}
  \end{column}
  \begin{column}{0.35\textwidth}
    \pgfimage[height=4cm]{./figs/archimedes}
  \end{column}
\end{columns}
\end{frame}
% -------------------------------------------------
% -------------------------------------------------
\begin{frame}
\frametitle{History: Method of Exhaustion}
\begin{itemize}
  \item A circle is not a polygon
  \item A circle {\bf is} a polygon with an infinite number of sides
  \item $C_n$ = circumference of an n-sided polygon inscribed in a circle of
  radius $1/2$
  \item $\lim_{n\rightarrow\infty} = \pi$
  \item Archimedes deterimined
    \begin{align*}
    \frac{223}{71} < & \pi < \frac{22}{7}\\
    3.1408 < & \pi < 3.1429
    \end{align*}
  \item two places of accuracy....
  \item \,\, \url{http://www.pbs.org/wgbh/nova/archimedes/pi.html}
\end{itemize}
\end{frame}
% -------------------------------------------------
% -------------------------------------------------
\begin{frame}
\frametitle{History: Method of Machin}
\begin{itemize}
  \item Around 1700, John Machin discovered the trig identity
  \begin{equation*}
  \pi  = 16 \arctan{(\frac{1}{5})} - 4\arctan{(\frac{1}{239})}
  \end{equation*}
  \item Led to calculation of the first 100 digits of $\pi$
  \item Uses the Taylor series of $\arctan$ in the algorithm
  \begin{equation*}
    \arctan{(x)} = x - \frac{x^3}{3} + \frac{x^5}{5} - \frac{x^7}{7}\dots
  \end{equation*}
  \item Used until 1973 to find the first Million digits
\end{itemize}
\end{frame}
% -------------------------------------------------
% -------------------------------------------------
\begin{frame}
\frametitle{Numerical Analysis}
\begin{definition}[Trefethen]
Study of algorithms for the problems of continuous mathematics
\end{definition}

We've been doing this since Calculus (and before!)
\begin{itemize}
  \item Riemann sum for calculating a definite integral
  \item Newton's Method
  \item Taylor's Series expansion + truncation
\end{itemize}
\end{frame}
% -------------------------------------------------
% -------------------------------------------------
\begin{frame}
\frametitle{Big Questions}
\begin{block}{}
\begin{itemize}
\item How algorithms work and how they fail
\item Why algorithms work and why they fail
\end{itemize}
\end{block}
\begin{itemize}
\item Connects mathematics and computer science
\item Need mathematical theory, computer programming, and scientific inquiry
\end{itemize}
\end{frame}
% -------------------------------------------------
% -------------------------------------------------
\begin{frame}
\frametitle{Numerical Analysis}
A Numerical Analyst needs
\begin{itemize}
\item computational knowledge (e.g. programming skills)
\item understanding of the application (physical intuition for validation)
\item mathematical ability to construct and meaningful algorithm
\end{itemize}
\end{frame}
% -------------------------------------------------
% -------------------------------------------------
\begin{frame}
\frametitle{Numerical Analysis}
Numerical focus:
\begin{description}
\item[Approximation] An approximate solution is sought.  How close is this
to the desired solution?
\item[Efficiency] How fast and cheap (memory) can we compute a solution?
\item[Stability] Is the solution sensitive to small variations in the
problem setup?
\item[Error] What is the role of finite precision of our computers?
\end{description}
\end{frame}
% -------------------------------------------------
% -------------------------------------------------
\begin{frame}
\frametitle{Numerical Analysis}
Why?
\begin{itemize}
\item Numerical methods improve scientific simulation
\item Some disasters attributable to bad numerical computing (Douglas
Arnold)
\begin{itemize}
\item The Patriot Missile failure, in Dharan, Saudi Arabia, on February 25, 1991 which resulted in 28 deaths, is ultimately attributable to poor handling of rounding errors.
\item The explosion of the Ariane 5 rocket just after lift-off on its maiden voyage off French Guiana, on June 4, 1996, was ultimately the consequence of a simple overflow.
\item The sinking of the Sleipner A offshore platform in Gandsfjorden near Stavanger, Norway, on August 23, 1991, resulted in a loss of nearly one billion dollars. It was found to be the result of inaccurate finite element analysis.
\end{itemize}
\end{itemize}
\end{frame}
% -------------------------------------------------
% -------------------------------------------------
\begin{frame}
\frametitle{I thought we were studying "Numerical Methods"?!}
\begin{block}{}
Numerical analysis is the study of numerical methods
\end{block}
\begin{itemize}
\item Numerical Analysis: understanding general behavior of numerical
methods
\item Numerical Methods: understanding {\bf how} to apply certain methods to
solve specific tasks
\item As programmers, we need to understand the concepts of numerical
analysis and implementation aspects of the numerical method
\end{itemize}

We thus focus on 
\begin{itemize}
\item Matlab implementation
  \begin{itemize}
  \item fast learning curve
  \item quick time-to-production: low development times
  \item a major develop environment in scientific computing
  \item integrated graphics
  \end{itemize}
\item Errors in computation
\item specific methods for root finding, integrating, interpolation, etc.
\end{itemize}
\end{frame}
% -------------------------------------------------
\begin{frame}
\frametitle{CS Topics}
We'll visit a variety CS areas during the semester:
\begin{description}
\item AI:  transitions, state systems, patterns (eigenvalues, linear algebra)
\item Informatics:  Google matrix, Amazon recommendations (eigenvalues, linear
algebra, sparse matrices, iterative methods)
\item Graphics:  image compression, representation of
curves/surfaces/lighting (interpolation, differentiation, etc)
\item Systems:  computer performance and monitoring (interpolation, integration)
\item Security:  ssh (random numbers)
\item Economics/Finance:  modeling/simulation of financial data (monte carlo)
\item Scientific computing:  design of algorithms for high performance/parallel computing.
\item ... 
\end{description}
\end{frame}
% -------------------------------------------------
% -------------------------------------------------
\begin{frame}
\frametitle{Course Info}
\begin{block}{}
\url{http://www.cs.uiuc.edu/class/fa09/cs357/}
\end{block}
\begin{columns}
\begin{column}{0.55\textwidth}
\begin{itemize}
\item Book: Numerical Mathematics and Computing, Sixth Edition, by
Cheney and Kincaid (or 5th)
\end{itemize}
\end{column}
\begin{column}{0.35\textwidth}
\pgfimage[height=2cm]{./figs/nmc}
\end{column}
\end{columns}

\bigskip
{\small
\begin{columns}
\begin{column}{0.32\textwidth}
Luke's contact:
\begin{itemize}
\item Office: 4312 Siebel Center
\item Office Hours: 10:45-11:45am
\item email: lukeo AT illinois.edu
\item IM: lukeo (jabber.cs.uiuc.edu) %to be created
\end{itemize}
\end{column}
\begin{column}{0.32\textwidth}
\begin{itemize}
\item TA: Jacob Schroder, Elena Caraba, Matt Wrobel
\item Office Hours: tbd
\item Office Hours Location: 4335 Siebel Center (NA Lab)
\item newsgroup: cs357 at news.cs.uiuc.edu
\end{itemize}
\end{column}
\end{columns}
}
\end{frame}
% -------------------------------------------------
% -------------------------------------------------
\begin{frame}
\frametitle{Assessment}
\begin{block}{}
\url{http://www.cs.uiuc.edu/class/fa09/cs357/}
\end{block}
Grades:
\begin{tabular}{l l}
Homework   & 25\%\\
Midterm Exam 1 & 25\%\\
Midterm Exam 2 & 25\%\\
Final Exam & 25\%\\
\end{tabular}

Homework: 
\begin{itemize}
\item must acheive 50\% to pass the course
\item weekly, due by 4pm at the TAs office (box)
\item submitted in paper, do not email
\item source code included upon request
\item no graphics by hand
\item accepted for 50\% credit up to one week late
\item no dropped scores
\item collaborate, but {\bf do not} copy!
\begin{itemize}
\item cite collaborators at the top
\item hand in your {\bf own} work in your {\bf own} words
\item copied solutions for 1/2 credit is still copying
\item see departmental policy re: cheating
\end{itemize}
\end{itemize}
\end{frame}
% -------------------------------------------------
% -------------------------------------------------
\begin{frame}
\frametitle{Finally$\ldots$}
\begin{block}{}
\url{http://www.cs.uiuc.edu/class/fa09/cs357/}
\end{block}
\vspace{1cm}
Schedule and Notes:
\begin{itemize}
\item on the web
\item everything is {\bf tentative} including midterm exams
\item final exam is fixed
\item ''Good Questions worksheets'' are lecture only
\end{itemize}
\vspace{1cm}
\begin{alertblock}{}
\centering
Questions?
\end{alertblock}
\end{frame}
% -------------------------------------------------
\begin{frame}
\frametitle{Preliminaries...}
\begin{itemize}
  \item A computational experiment...
  \item relative vs absolute error
  \item Taylor
  \item nested multiplication
  \item Matlab
\end{itemize}
\begin{block}{Goals}
  \begin{itemize}
  \item To think like a numerical analyst
  \item Understanding error in numerical computing
  \item A look at the power of Tayor Series
  \item A brief look at the algorithmic importance in numerical methods
  \item Utilizing Matlab
\end{itemize}
\end{block}
\end{frame}
% -------------------------------------------------
\begin{frame}
\frametitle{Typical Process}
\begin{block}{Goal}
  Find $f'(1.0)$ for function $f(x)=e^{x}$.
\end{block}
\begin{block}{Problem}
  Don't (want to need to) know $f'(x)$ explicitly.
\end{block}
\begin{block}{Method}
  Use definition.
  \begin{equation*} 
    f'(x) = \lim_{h\rightarrow 0} \frac{f(x+h) - f(x)}{h}
\end{equation*}
\end{block}
\end{frame}
% -------------------------------------------------
% -------------------------------------------------
\begin{frame}[fragile]
\begin{block}{Method}
  Use definition.
  \begin{equation*} 
    f'(x) = \lim_{h\rightarrow 0} \frac{f(x+h) - f(x)}{h}
\end{equation*}
\end{block}
\frametitle{Typical Process continued}
\begin{columns}
\begin{column}{0.45\textwidth}
\begin{lstlisting}[mathescape,caption=psuedo]

$h$=1
$x$=1  
for $i$=1 to 20 do
  $h=h/2$
  $y=(f(x+h)- f(x))/h$
  $err = |(f(x)-y)|$
end
\end{lstlisting}
\end{column}

\begin{column}{0.45\textwidth}
\begin{lstlisting}[caption=Matlab]
f=inline('exp(x)');
h=1;
x=1;
for i=1:20
  h=h/2;
  y=(f(x+h) - f(x))/h;
  err(i) = abs(f(x)-y);
end
\end{lstlisting}
\end{column}
\end{columns}
\end{frame}
% -------------------------------------------------
% -------------------------------------------------
\begin{frame}
\frametitle{Measuring Error}
  \begin{itemize}
    \item Here we used the absolute error:
  \begin{align*}
    \text{Error}(x_a) &= |x_t - x_a|\\
  & = |\text{true value} - \text{approximate value}|
  \end{align*}

    \item This doesn't tell the whole story.  For example, if the values
are large, like billions, then an Error of 100 is small.  If the values
are smaller, say around 10, then an Error of 100 is large.  We need the
relative error:
  \begin{align*}
    \text{Rel}(x_a) &= \left|\frac{x_t - x_a}{x_t}\right|\\
  & = \left|\frac{\text{true value} - \text{approximate value}}{\text{true
value}}\right|
  \end{align*}
  \item More on \emph{types} of errors next time...
\end{itemize}
\end{frame}
% -------------------------------------------------
% -------------------------------------------------
\begin{frame}
\frametitle{Taylor}
  \begin{itemize}
    \item All we can ever do is add and multiply (more on the FPU later)
    \item We can't directly evaluate $e^{x}$, $cos(x)$, $\sqrt{x}$
    \item What to do?  Taylor Series \emph{approximation}
\end{itemize}
\begin{block}{Taylor}
  The Taylor series expansion of $f(x)$ at the point $x=c$ is given by
  \begin{align*}
    f(x) &= f(c) + (x-c) f'(c) + \frac{(x-c)^2}{2!} f''(c) + \dots +
\frac{(x-c)^n}{n!} f^{(n)}(c)+\dots\\
        &= \sum_{k=0}^{\infty} \frac{(x-c)^{k}}{k!} f^{(k)}(c)
\end{align*}
\end{block}
\end{frame}
% -------------------------------------------------
% -------------------------------------------------
\begin{frame}
\frametitle{Taylor Example}
\begin{block}{Taylor}
  The Taylor series expansion of $f(x)$ about the point $x=c$ is given by
  \begin{align*}
    f(x) &= f(c) + (x-c) f'(c) + \frac{(x-c)^2}{2!} f''(c) + \dots +
\frac{(x-c)^n}{n!} f^{(n)}(c)+\dots\\
        &= \sum_{k=0}^{\infty} \frac{(x-c)^{k}}{k!} f^{(k)}(c)
\end{align*}
\end{block}
\begin{example}[$e^x$]
    We know $e^0 = 1$, so expand about $c=0$ to get
    \begin{align*}
    f(x) & = e^x = 1 + (x-0)\cdot 1 + \frac{(x-0)^2}{2~} \cdot 1 + \dots\\
          & = 1 + x + \frac{x^2}{2!} + \frac{x^3}{3!} + \dots
\end{align*}
\end{example}
\end{frame}
% -------------------------------------------------
% -------------------------------------------------
\begin{frame}
\frametitle{Taylor Approximation}
\begin{itemize}
\item
    So
    \begin{equation*}
    e^2 = 1 + 2 + \frac{2^2}{2!} + \frac{2^3}{3!} + \dots
\end{equation*}
\item But we can't evaluate an infinite series, so we truncate...
\end{itemize}
\begin{block}{Taylor Series Polynomial Approximation}
  The Taylor Polynomial of degree $n$ for the function $f(x)$ about the
point $c$ is 
\begin{equation*}
        p_n(x) = \sum_{k=0}^{n} \frac{(x-c)^{k}}{k!} f^{(k)}(c)
\end{equation*}
\end{block}
\begin{example}[$e^x$]
In the case of the exponential
\begin{equation*}
        e^x \approx p_n(x) = 1 + x + \frac{x^2}{2!} + \dots + \frac{x^n}{n!}
\end{equation*}
\end{example}
\end{frame}
% -------------------------------------------------
% -------------------------------------------------
\begin{frame}[fragile]
\frametitle{Taylor Approximation}
Evaluate $e^2$:
\begin{itemize}
  \item Using $0^{th}$ order Taylor series: $e^x \approx 1$ does not
give a good fit.
  \item Using $1^{st}$ order Taylor series: $e^x \approx 1 + x$ gives a
better fit.
  \item Using $2^{nd}$ order Taylor series: $e^x \approx 1 + x + x^2/2$ gives a
a really good fit.
\end{itemize}
\begin{lstlisting}
x=2;
pn=0;
for j=0:15
  pn = pn + (x^j)/factorial(j);
  err = exp(2)-pn
  pause
end
\end{lstlisting}

\bigskip
Interactive module: http://www.cse.uiuc.edu/iem/interpolation/taylor/
\end{frame}
% -------------------------------------------------
% -------------------------------------------------
\begin{frame}
\frametitle{Taylor Approximation Recap}
\begin{block}{Infinite Taylor Series Expansion (exact)}
\begin{equation*}
    f(x) = f(c) + (x-c) f'(c) + \frac{(x-c)^2}{2!} f''(c) + \dots +
\frac{(x-c)^n}{n!} f^{(n)}(c)+\dots
\end{equation*}
\end{block}
\begin{block}{Finite Taylor Series Expansion (exact)}
\begin{equation*}
    f(x) = f(c) + (x-c) f'(c) + \frac{(x-c)^2}{2!} f''(c) + \dots +
\frac{(x-c)^n}{n!} f^{(n)}(\xi),
\end{equation*}
but we don't know $\xi$.
\end{block}
\begin{block}{Finite Taylor Series Approximation}
\begin{equation*}
    f(x) \approx f(c) + (x-c) f'(c) + \frac{(x-c)^2}{2!} f''(c) + \dots +
\frac{(x-c)^n}{n!} f^{(n)}(x),
\end{equation*}
\end{block}
\end{frame}
% -------------------------------------------------
% -------------------------------------------------
\begin{frame}
\frametitle{Taylor Approximation Error}
\begin{itemize}
  \item How accurate is the Taylor series polynomial approximation?
  \item The $n$ terms of the approximation are simply the first $n$
terms of the \emph{exact} expansion:
  \begin{equation}
    e^x = \underbrace{1 + x + \frac{x^2}{2!}}_{\text{$p_2$ approximation
to $e^x$}} + \underbrace{\frac{x^3}{3!} + \dots}_{\text{truncation
error}}
\end{equation}
\item 
So the function $f(x)$ can be written as the Taylor Series approximation
plus an error (truncation) term:
\begin{equation*}
  f(x)  = f_n(x) + E_n(x)
\end{equation*}
where
\begin{equation*}
  E_n(x) = \frac{(x-c)^{n+1}}{(n+1)!}f^{(n+1)}(\xi)
\end{equation*}
\end{itemize}
\end{frame}
% -------------------------------------------------
% -------------------------------------------------
\begin{frame}
\frametitle{Big "$\mO$" (different view)}
We write the error as 
\begin{align*}
  E_n(x) &= \frac{(x-c)^{n+1}}{(n+1)!}f^{(n+1)}(\xi)\\
  & = \mO\left(\frac{(x-c)^{n+1}}{(n+1)!}\right)
\end{align*}
since we assume the $(n+1)^{th}$ derivative is bounded on the interval
$[a,b]$.

\vspace{0.5cm}
Often, we let $h=x-c$ and we have
\begin{equation*}
  f(x) = p_n(x) + \mO(h^{n+1})
\end{equation*}

\end{frame}
% -------------------------------------------------
% -------------------------------------------------
\begin{frame}
\frametitle{Truncation Error}
The Taylor series expansion of $\sin{(x)}$ is 
\begin{equation*}
  \sin{(x)} = x 
  -\frac{x^3}{3!}
  +\frac{x^5}{5!}
  -\frac{x^7}{7!}
  +\frac{x^9}{9!} -\dots
\end{equation*}
If $x \ll 1$, then the remaining terms are small.  If we neglect these terms
\begin{equation*}
  \sin{(x)} = \underbrace{x 
  -\frac{x^3}{3!}
  +\frac{x^5}{5!}}_{\text{approximation to sin}}
  \underbrace{-\frac{x^7}{7!}
  +\frac{x^9}{9!} -\dots}_{\text{truncation error}}
\end{equation*}
\end{frame}
% -------------------------------------------------
% -------------------------------------------------
\begin{frame}
\frametitle{Outlook}
Next time:
\begin{itemize}
  \item Polynomial evaluation
  \item Matlab
\end{itemize}
\bigskip
Overall:
\begin{enumerate}
  \item Preliminaries
  \item Floating point arithmetic, precision, etc
  \item Nonlinear equations (root finding)
  \item Interpolation, Differentiation, Integration, splines
  \item Midterm Exam 1

  \item Numerical Linear Algebra
  \item Midterm Exam 2

  \item Application: ODEs, more splines, more linear algebra, Monte Carlo
  \item Final Exam
\end{enumerate}
\end{frame}
% -------------------------------------------------
\end{document}
